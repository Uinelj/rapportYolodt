\section{Manuel utilisateur}
%\label{p2}
Le programme est utilisable en deux versions : Une première non-interactive en ligne de commande, et une seconde interactive graphique.
\subsection{Interface en ligne de commande}

Le programme peut être invoqué avec 3 arguments différents. L'abscence de arguments renvoie une erreur et un rappel sur la syntaxe des arguments attendus.

-d <folder1> <folder2> ... : Ajoute les dossiers folder1 folder2 ... à la base de données, c'est-à-dire recherche récursivement tous les ODT présents dans ces dossiers, extrait et analyse leur contenu et enregistre les titres pour une recherche future.

-f <file> : Affiche les informations importantes de l'ODT file : Chemin, titres.

-w <words> : Retourne les ODT les plus pertinents en fonction des mots recherchés. Si un "+ " est envoyé, la recherche se fera en mode ET (ie. on cherchera un ODT contenant tous les mots spécifiés après le +). Si un ensemble de mots est spécifié entre guillemets, le programme ne retournera que des ODT contenant exactement la phrase entre guillemets.