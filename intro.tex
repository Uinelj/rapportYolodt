\section{Introduction} % Numérotation
%\addcontentsline{toc}{section}{Introduction} % Ajout dans la table des matières

\subsection{Contexte}
Nous sommes en Licence 2 Mathématiques-Informatique à l'Université de Cergy-Pontoise. Parmi les enseignements dispensés au cours du premier semestre se trouve un module de programmation orientée objet (POO), en Java. C'est dans ce contexte d'introduction à la POO que nous a été proposé le projet ( baptisé Yolodt ) dont il est question.
\subsection{Objectif}
Yolodt est un programme qui a pour objectif de proposer une indexation de fichiers de type "Open Office Document", d'un stockage de leur titres, puis d'une recherche ultérieure d'informations contenues dans ceux-ci.
\subsection{Environnement}
Le projet a été développé avec le Java Development Kit 7 (JDK 7), à l'aide de l'Environnement de Développement Integré (IDE) Eclipse Luna. 
Les systèmes d'exploitation utilisés durant toute la durée du développement ont été Debian, ArchLinux et ElementaryOS. 
Ces trois systèmes sont de type Linux.
\subsection{Composition}
Le groupe est composé de Julien Abadji et Bastien Lepesant. Nous avons des façons différentes de réfléchir, et nous avons pensé intéressant de nous mettre en binôme afin de confronter nos idées souvent divergentes, dans le but de toujours s'approcher de la conception optimale du projet.  %De plus, ceci nous a permis de mesurer l'importance de la communication dans un projet à plusieurs : en effet, sans une quasi constante remise en question des idées blahblah pour la conclusion.